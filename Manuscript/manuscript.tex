\documentclass[12pt, letterpaper, titlepage]{article}
\usepackage[utf8]{inputenc}

%% preamble
\usepackage[margin = 1in]{geometry}
\usepackage{booktabs}
\usepackage{natbib}
\usepackage[colorlinks=true, citecolor=blue]{hyperref}
\usepackage{graphicx}
\usepackage{subcaption}

%% metadata
\title{An Analysis of the Use of Average Centipawn Loss to Detect Cheating in Chess}

\author{James Horbury\\
    University of Connecticut
}
\date{\today}

\begin{document}
\maketitle

\begin{abstract}
\section*{Abstract}
\addcontentsline{toc}{section}{Abstract}
\label{sec:abs}

Evolutions in chess computing have allowed for extensive libraries of past games to be analyzed again utlizing centipawn math. A recent scandal involving the alleged cheating of Chess Grandmaster Hans Niemann in the sphere of professional chess has launched a conversation about how cheaters can be detected statistically to the forefront of Academic Data Science. This paper will explore how one might detect cheating using average centipawn loss and whether these methods are reliable enough to become a standard in the chess proctors handbook.

\bigskip
\noindent\sc{Kewords}:
Centipawn math; 
Niemann;
Carlsen;
Chess;
\end{abstract}

\section*{Introduction}
\addcontentsline{toc}{section}{Introduction}
\label{sec:intro}

Over the past few months, the global chess community has been shaken by a very high-profile story regarding Hans Niemann, Magnus Carlsen,  and “alleged cheating” in the world of professional chess. The scandal started when Carlsen, the currently reigning five-time World Chess Champion, unexpectedly withdrew from the Sinquefield Cup after a stunning upset by the then nineteen-year-old Niemann in the tournament's third round. This was the first time in Carlsen's career where he withdrew from a major event in progress, and this fact alone was enough to cause a stir among observers.

Carlsen also didn't immediately state the reason as to why he withdrew at first, though many correctly assumed he was holding his tongue in making any sort of public accusation of cheating to avoid a retaliatory lawsuit by Niemann while he didn't have proof. But it was too late, the public was already running wild with the story and amateur and professional entities alike were launching their own investigations on the matter, looking for any indication of foul play.

Although the question of whether Hans Niemann “cheated” originally stemmed from just the one game, it would be nearly impossible to prove foul play by simply analyzing moves without any form of “smoking gun” in terms of evidence, such as the discovery of a concealed communication device or computer on his person. This is primarily due to the modality in which the game was played.

The game between Carlsen and Niemann was played “over the board” (OTB) in contrast to “virtual chess” which is hosted online. As one might guess cheating in online chess is quite easy. One simply has to rely on outside input, commonly through a chess engine (such as Stockfish, AlphaZero, and MuZero) and execute the moves from their game into the engine as they played, following up with whichever move that was suggested to play next. This has become an ever increasing problem as the superiority of top chess engines have been acknowledged ever since IBM's Deep Blue defeated former World Chess Champion Garry Kasparov in 1997.

But, cheating OTB is not as simple. Players are commonly required to pass through a metal detector and are not permitted to bring phones or any other communication devices with them, making it exceedingly difficult for Niemann to have cheated this way. The specific allegations as to how Niemann could have cheated OTB were and currently are still unknown. However, since none were found prior, any possibility of this has already been eliminated.

One may think the matter would be closed at that, but not quite. Maybe we can't divine anything from a single game, but if a trend could somehow be established then, at the very least Carlsen's suspicions could be validated. That is to say, if the investigation were to propagate into Niemann's history, where statistical analysis would be possible given the amount of data, one might be able to identify suspicious activity. The data is there and Carlsen still holds the public trust, all that needs to be worked out is the how.
Chess.com, which is used by millions of players, has released an expose on their cheat detection methods, specifically with regard to Hans Niemann. The methods used include (1) comparing moves made to engine-recommended moves, (2) comparing player past performances to their historical strength, (3) comparing player's performance to comparable peers, (4) looking at behavioral factors, and (5) reviewing the time usage for finding “easy” vs “difficult” moves.

Item 2 requires an assessment of move quality which is usually done in centipawns. The centipawn is a unit of measure used in chess as a representation of advantage. One centipawn is equal to one hundredth of a pawn and, although its value plays no formal role in the game, they are used by chess engines and players alike as a means to evaluate the positions of a chess board via centipawn loss. Centipawn loss (CL) is a calculation and numerical score given by a chess engine  (Stockfish, Houdini, Koodo, etc.) as the difference, in centipawns between the move an individual makes and the “strongest” move available at the time as computed by the engine.
In practice, ACL is usually somewhere between 0 and 100, where the closer a player is to 0 the better their moves have been. A score of 100 means one is losing the equivalent of one pawn per move which would generally be hard to maintain without quickly being mated (unless your opponent misses every chance they get).

\section*{Data}
\addcontentsline{toc}{section}{Data}
\label{sec:data}

The dataset was compiled by Rafael Vicente Leite, an independent chess researcher and data scientis, immediately following the events of the scandal with the intent of conducting his own investigation of the Niemann controversy. The data itself was sourced from ChessBase, a German chess software company that maintains and sells large-scale databases containing the moves of recorded professional chess games. In addition to just holding data from prior games, however, it also provides some engine analysis of these games using Stockfish. Liete converted this raw data, that is to say the archive of games, from their initial portable game notation (PGN) format to data that was more amenable to statistical analysis.

Specifically, the processed dataset is comprised of ranked games (both virtual and OTB) of the following professional chess players: Hans Niemann, Erigaisi Arjun, Gukesh D, Alireza Firouzja, Vincent Keymer, and Igors Rausis.
% I don't know the exact reasoning for why these players were selected aside from Niemann and Carlsen, the two players who sparked the cheating controversy, and Rausis, a known cheater and perhaps "control" of his analysis.

It is a comprehensive list of every professional game they have played from 2018 to 2022 and is provided in both .csv and .pkl format. Some of the relevant information it contains are what color was played, what their respective ELO scores were at the time, and calculated ACL scores were along with some other unimportant details for this analysis (e.g., ACL score of Niemann's opponent, tournament name and event round, etc.). The main information of concern with regard to the entirety of the data is Niemann's ELO score and calculated ACL score of each game of each particular point in time, as this is what will be used to plot. All other games that included information where Niemann was not playing were dropped from the original dataset provided by Lichess.

\section*{Methods}
\addcontentsline{toc}{section}{Methods}
\label{sec:sim}

% refactor dataset before adding back methods

\section*{Application}
\addcontentsline{toc}{section}{Application}
\label{sec:app}

% removed faulty figures

\section*{Discussion}
\addcontentsline{toc}{section}{Discussion}
\label{sec:disc}

% summarize contributions of research and future directions
The results of this research largely confirm the claims included in Chess.com's report on Niemann however, although the data does indicate suspicious activity, this is not a "cut and dry" case of cheating, so to speak. Observing a single method to detect cheating as we have through average centipawn loss is not enough to confirm for sure that Hans Niemann cheated, both virtually and OTB, more than he claimed but this was never truely the objective of this research. The real question posed here was whether historical records of average centipawn loss compared to his ELO ranking could be treated as a "smoking gun" or surefire way to prove cheating alone; which it does not. Note: Relevant figures not available, fix this later. 

\bibliographystyle{chicago}
\bibliography{citations}

\end{document}