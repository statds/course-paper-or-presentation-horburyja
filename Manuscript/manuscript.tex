\documentclass[12pt]{article}
\usepackage[utf8]{inputenc}

%% preamble
\usepackage[margin = 1in]{geometry}
\usepackage{booktabs}
\usepackage{natbib}
\usepackage[colorlinks=true, citecolor=blue]{hyperref}

%% metadata
\title{An Analysis of the Use of Centipawn Loss to Detect Cheating in Chess}
\author{James Horbury\\
    University of Connecticut
}
\date{November 14, 2022}

\begin{document}
\maketitle

\section*{Abstract}
\addcontentsline{toc}{section}{Abstract}
\label{sec:abs}

% revisit when research is completed

\section*{Introduction}
\addcontentsline{toc}{section}{Introduction}
\label{sec:intro}

Over the past few months, the global chess community has been shaken by a very high-profile story regarding Hans Niemann, Magnus Carlsen, and “alleged cheating” in chess. This “cheating accusation” scandal started when Carlsen withdrew from the Sinquefield Cup after losing to nineteen-year-old Niemann in a stunning third round defeat. This was the first time in the then reigning five-time World Chess Champion's career where he withdrew from a major event in progress.

% Go into detail about why this was such a shock. Include backgrounds of Carlsen and Niemann.

Carlsen didn't immediately state the reason as to why he withdrew at first, though many correctly assumed that he was holding his tongue in making any sort of public accusation to avoid a potential retaliatory lawsuit for slander.

It should be noted that Carlsen played Niemann again two weeks later in the Julius Baer Generation Cup, but resigned after his second move, perpetuating rumors of the scandal even further. 
This has become a matter of significant speculation both inside and outside the chess world given the nature of the situation and the people involved. 

The first game was played “over the board” (OTB) in contrast to “virtual chess” which is hosted online. Cheating in online chess is very easy. One simply needs to run a chess engine (like Stockfish) and execute the moves from their game into the engine as they played, following up with whichever move it suggested they play next. Cheating OTB is not as simple. Players are commonly required to pass through a metal detector and are not permitted to bring phones or any other electronic devices with them (see the afore mentioned article about the famous concealed shoe computer).

The specific allegations related to how Niemann cheated OTB are unknown. Also, it would be very hard to prove cheating by simply looking at moves that are made in a single game. But what if you had access to hundreds or even thousand of games from a certain individual over a long stretch of their career?
Although cheating in online chess should be, in principle, at least as hard to detect as with cheating OTB, with the added disadvantage that you cannot enforce the disallowance of electronic devices, in reality it is quite different. Chess.com, which is used by millions of players has released an expose on their cheat detection methods, specifically with regard to Hans Niemann, who they allege has cheating in many games. The methods used include (1) comparing moves made to engine-recommended moves, (2) comparing player past performances to their historical strength, (3) comparing a player's performance to comparable peers, (4) looking at behavioral factors, and (5) reviewing time usage for finding “easy” vs. “difficult” moves.  

\section*{Data}
\addcontentsline{toc}{section}{Data}
\label{sec:data}

% insert data from lichess

\section*{Methods}
\addcontentsline{toc}{section}{Methods}
\label{sec:meth}

\section*{Simulation}
\addcontentsline{toc}{section}{Simulation}
\label{sec:sim}

% push code from local repo

\section*{Application}
\addcontentsline{toc}{section}{Application}
\label{sec:app}

\section*{Discussion}
\addcontentsline{toc}{section}{Discussion}
\label{sec:disc}

\section*{Appendix}
\addcontentsline{toc}{section}{Appendix}
\label{sec:appx}

\section*{References}
\addcontentsline{toc}{section}{References}
\label{sec:refs}

\end{document}