\documentclass[12pt]{article}
\usepackage[utf8]{inputenc}

%% preamble
\usepackage[margin = 1in]{geometry}
\usepackage{booktabs}
\usepackage{natbib}
\usepackage[colorlinks=true, citecolor=blue]{hyperref}

%% metadata
\title{An Analysis of the Use of Centipawn Loss to Detect Cheating in Chess}
\author{James Horbury\\
    University of Connecticut
}
\date{October 10, 2022}

\begin{document}
\maketitle

\section*{Introduction}
\addcontentsline{toc}{section}{Introduction}
\label{sec:intro}

Recently there has been a very high profile case of alleged cheating in the world of professional chess 
where the current reigning world chess champion, GM Magnus Carelson, was defeated by GM Hans Niemann, a player 
that was ranked 42nd in the world and experineced massive growth in FIDE rating in the past two years. This 
has caused many skeptics in the chess community, including Carelson himself, to suspect cheating may have been 
at play. The ensuing comotion has caused many hobbiests to stop and consider how pervasive cheating
really is in chess, more noteably in online venues, and what methods are employed to flag games with alleged 
engine-aided moves. This paper will surve as an analysis of the currently available methods of cheat detection 
and for which cases they are more or less robust.

Current research \citep{regan2011understanding} has been done by Dr. Kenneth R. Regan, of the Computer Science and Engineering 
Department of the University at Buffalo, employing the use of a model that utilizes the Jenson-Shannon 
Divergence as a basis for detecting player vs. engine games. However, there is much scrutiny over the 
transparency of his methods and their effectiveness on different "intensities" of cheating. There are also 
other \href{https://github.com/MGleason1/PGN-Spy}{open-source options} that can provide an indepth analysis of move-for-move comparisons
in specific games, an area where Dr. Regan's model appears to be weak.

\section*{Specific Aims}
\addcontentsline{toc}{section}{Specific Aims}
\label{sec:spec}

The most widespread methods of cheat detection rely on large amounts of data from the past games of specific 
players, such as graphing their average centipawn loss per game against their historic FIDE rating. What other 
alternative methods of cheat detection can be reliably used where the norm falls short?

The purpose of posing this question is to expose the inner-workings of the methodologies behind dectecting 
cheating in chess in the digital age and provide scientific reasoning to support accusations of cheating in 
instances where they are valid.

\section*{Data Description}
\addcontentsline{toc}{section}{Data Description}
\label{sec:data}

I'll be utilizing a dataset of Data for 20,000+ chess games played on Lichess \citep{mysarahmadbhat2021} 
(including moves, victor, rating, opening details, etc.) to provide insight into how centipawn loss relates 
to FIDE score in the general population, then do the same using historical data for GM Magnus Carelson 
\citep{zaidqureshi2022carlsen} and GM Hans Niemann \citep{zaidqureshi2022niemann} to observe any differences. These are user 
submitted examples through Kaggle, but I should be able to have the same access to the games through the Lichess website.

\section*{Research Design/Methods/Schedule}
\addcontentsline{toc}{section}{Research Design/Methods/Schedule}
\label{sec:res}

To reitterate, I plan to plot average centipawn loss per game vs. FIDE score for both the general population 
and for the GM's in question to illustrate the common view that this is a good indicator of cheating in chess. If 
other examples can be found that follow a growth pattern similar to Niemann's I'll present them also to offer a 
fair counterargument to the claim that his success is "unnatural". Then, using the opensource move-for-move comparison 
software mentioned in the Introduction I can go more in depth into how, if possible, Niemann could/could not have 
cheated in his game against Carlsen. Regarding specific metrics, I'll have to see what the PGN-Spy actually offers in 
terms of applicable output data and take my analysis from there, which will require more research than what was done 
for this proposal. Additionally some data cleaning/preparation will be necessary for each Lichess datasets to ensure 
accuracy.

\section*{Discussion}
\addcontentsline{toc}{section}{Discussion}
\label{sec:disc}

I expect to find that there may be reason to belive that the growth Niemann experience could be considered 
not outside the bounds of normal given the centipawn loss vs. FIDE score method and that a more in-depth
analysis must be conducted using other methods of measurement. This would challenge the existing assumptions that
centipawn loss is a "smoking gun" for cheating and that there are instances it is inaccurate.

\section*{Conclusion}
\addcontentsline{toc}{section}{Conclusion}
\label{sec:conc}

In summary, the goal of this proposal is to analyze the use of comparing centipawn loss to FIDE rating as a metric 
to detect cheating in chess, and whether move-for-move analysis can support the claim that Niemann likely cheated 
in his game against Carlsen.

\bibliographystyle{chicago}
\bibliography{citations}

\end{document}