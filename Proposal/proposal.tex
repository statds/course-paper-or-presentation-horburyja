\documentclass[12pt]{article}
\usepackage[utf8]{inputenc}

%% preamble
\usepackage[margin = 1in]{geometry}

%% metadata
\title{An Analysis of Engine vs. Player Detection in Online Chess using the Jensen-Shannon Divergence}
\author{James Horbury\\
    University of Connecticut
}
\date{October 10, 2022}

\begin{document}
\maketitle

\section*{Introduction}
\addcontentsline{toc}{section}{Introduction}
\label{sec:intro}

Recently there has been a very high profile case of alleged cheating in the world of professional chess 
where the current reigning world chess champion, GM Magnus Carelson, was defeated by GM Hans Niemann, a player 
that was ranked 42nd in the world and experineced massive growth in FIDE rating in the past two years. This 
has caused many skeptics in the chess community, including Carelson himself, to suspect cheating may have been at
play. The ensuing comotion has caused many hobbiest, like myself, to stop and consider how pervasive cheating
really is in chess, more noteably in online venues, and what methods are employed to flag engine-aided moves.

Current research (link) is being done by Dr. Kenneth R. Regan, of the Computer Science and Engineering 
Department of the University at Buffalo, employing the use of the Jenson-Shannon Divergence as a measure of 
similarity between two distributions.

\section*{Specific Aims}
\addcontentsline{toc}{section}{Specific Aims}
\label{sec:spec}

Specifically, I would like to investigate how accurate this method of cheat detection is when using a dataset 
of engine vs. player games as opposed to player vs. player games. This leads me to question the standard at 
which cheating in chess is measured and whether accusations/bans of alleged cheating are issued justly. I chose 
to conduct an analysis of this method of cheat detection to promote transparency in this regard.

\section*{Data Description}
\addcontentsline{toc}{section}{Data Description}
\label{sec:data}



\section*{Research Design/Methods/Schedule}
\addcontentsline{toc}{section}{Research Design/Methods/Schedule}
\label{sec:res}

\section*{Discussion}
\addcontentsline{toc}{section}{Discussion}
\label{sec:disc}

\section*{Conclusion}
\addcontentsline{toc}{section}{Conclusion}
\label{sec:conc}

\bibliographystyle{chicago}
\bibliography{citations}

\end{document}